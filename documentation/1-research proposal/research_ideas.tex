\documentclass{article}
\usepackage{hyperref}
\usepackage[a4paper, total={5in, 9in}]{geometry}

\title{Research Proposal}

\author{Camila Blanes}
\date{}
\begin{document}
	\maketitle

\subsubsection*{Motivation}

Reelection is one of the principal mechanisms to hold accountable politicians in a government. This mechanism allows voters to remove legislators from office if they deviate too far from their preferences. The central theoretical literature of reelection and career concerns is given by the theory of political agency, which show us that politicians will be interested in voters' preferences as long as they help them achieve their career concerns. As a result, office motivated politicians have clearly incentives to be responsive to their constituency and cultivate a personal vote. \\
\\
Most literature and empirical evidence on career concerns and accountability is focused on legislative settings where there is long-term reelection incentives and electoral procedures are candidate-centered, such as the case of United Sates. However in institutional settings where elections are centered around parties, motivations will be reflected by a different behavior. 

\subsubsection*{Research Question}
 Mexico has a mixed method electoral system, members are elected to the chamber by proportional representation or single-seat districts. The way a representative gets his or her seat, creates a different relationship with their constituency and  with the party. This institutional setting generates different legislative behavior across type of legislators. Mexico has elections that are centered around parties and therefore the institutional setting produce low incentives to cultivate a personal base. The Electoral Reform of 2014 introduced a series of laws that changed the institutional political-electoral framework. Among the most relevant is the introduction of reelection for local and federal legislators. The main purpose of the reform was to improve politicians’ responsiveness to electoral preferences, professionalization of the parliamentary exercise and to improve legislative productivity (Reforma Político-Electoral (2014)).  \\
 \\
 Although the Electoral Reform of 2014 forces politicians to be re-elected through parties, that is the party has to approve the candidacy for reelection, the voters became a new principal to the representatives, who will also decide who stays in office. This research aims to leverage the systematic differences of type of legislators in the Chamber od Deputies and estimate the effect of on legislative behavior of both types of legislators.\\
 \\
  Although the Electoral Reform of 2014 forces politicians to be re-elected through parties, it is still not clear how the realtionship between members and parties changed. On one side, single-seat district members have more incentives cultivate a personal base and parties can get electoral advantage on the popularity of this legislators by obtaining more PR members. Members under the SSD system may fall under a conflicting scenario in which they have to deviate from party preferences to get closer to their constituency's preferences. The electoral reform affects in a different way the incentives of legislators, and therefore their behavior 
 
\subsubsection*{Data}
This research introduces a large new dataset on Mexican federal deputies. The dataset is composed by different variables that characterize legislative activity of members elected between 2006 and 2018. The first source of data captures legislative productivity. To construct this index, it is considered the full range of variables proposed by Dal Bó and Rossi (2011) that includes floor attendance, committee participation, time on floor speech, bills introduced, and bills ratified. Additionally, the dataset includes rolll call votes and text of each member's floor speech to capture policy position and party discipline. 

\subsubsection*{Descriptive Statistics}
\begin{table}[!htbp] \centering 
	\caption{Summary Statistics}
	\begin{tabular}{@{\extracolsep{4pt}} ccccc} 
		\\[-1.8ex]\hline 
		\hline \\[-1.8ex] 
		& Variable & Long Track  & Short Track & Difference of means  \\ 
		\hline \\[-1.8ex] 
		& Floor attendance (\%) & 0.897  & 0.819 &  0.0775** \\ 
		& &   &  &  (0.00478 ) \\ 
		& Number of Committees   &  2.77   &2.62 &  0.151**  \\ 
		& &   &  & (0.04649)  \\ 
		& Number of floor speeches   &  18.392 &16.649  &  1.74 \\ 
		& &   &  &  (1.266418) \\ 
		& Number of bills introduced  &  10.8  & 10.0 &  0.756 \\ 
		& &   &  & (0.6968)  \\ 
		& Vote with Party (\%)   &  0.842  & 0.759   &  0.0824** \\ 
		& &   &  &   (0.0053)\\ 
		\hline \\[-1.8ex] 
	\end{tabular} 
	\\
	\textit{Notes:} Standard errors are in parentheses. Short track corresponds to legislators that served from Legislatura LX to LXIII (Pre Electoral Reform). The number of observations is 2500. Significant at the 5\% level 
\end{table} 

\end{document}
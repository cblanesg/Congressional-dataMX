\documentclass{article}
\usepackage{hyperref}
\usepackage[a4paper, total={5in, 9in}]{geometry}

\title{Summary Statistics}

\author{Camila Blanes}
\date{}
\begin{document}
\section*{Summary Statistics: Collection of Data}

In order to study the legislative behavior of legislators in the Chamber of Deputies, I collected a dataset containing different variables that reflect legislative effort. The dataset includes five different terms (\textit{Legislaturas LX, LXI, LXII, LXIII, LXIV}). It is considered the full range of variables proposed by Dal Bó and Rossi, among data of roll call votes and floor speeches to obtain measures of ideology positioning and party discipline. This section provides an overview of the data and how it was collected. \\
\\
The following table summarizes the collected data. There are 500 available seats for each term in the Chamber. If a legislator withdrew after taking protest, a substitute takes the place, however for this exercise the legislative activity of the substitute was not considered.
\begin{table}[!htbp] \centering 
	\caption{Summary Statistics: Collection of Data}
	\begin{tabular}{@{\extracolsep{6pt}} ccccccc} 
		\\[-1.8ex]\hline 
		\hline \\[-1.8ex] 
		& Variable & Mean  & SD & Min  & Max & N \\ 
		\hline \\[-1.8ex] 
		& Number of Committees & 2.651  & 1.458 & 0  & 11 & 2500 \\ 
		& Floor Speech Interventions & 16.97 & 34.7 & 0  & 923 & 2450 \\ 
		& Floor Attendance & 161 & 53.20 &  1 &  210 & 2500 \\ 
		& Roll Call Votes & 568.6 & 173.0491 &  1 &  800 & 2500 \\ 
		& Bills Proposed & 10 & 14.49 &  1 &  284 & 2500 \\ 
		\hline \\[-1.8ex] 
	\end{tabular} 
	\\
	\textit{Unit of observation: legislator}
\end{table} 

\subsubsection*{Descriptive Statistics}
The following tables contain descriptive statistics in observables across type of legislators (RP vs SSD) and under type of electoral rules (Pre- Post Electoral Reform). \textit{Legislaturas} LX, LXI, LXII and LXIII fall under the period before the Electoral Rule, there are no reelection incentives. \textit{Legislatura} LXIV corresponds to legislators that have reelection incentives.   
\begin{table}[!htbp] \centering 
	\caption{Electoral Reform}
	\begin{tabular}{@{\extracolsep{4pt}} ccccc} 
		\\[-1.8ex]\hline 
		\hline \\[-1.8ex] 
		& Variable & Long Track  & Short Track & Difference of means  \\ 
		\hline \\[-1.8ex] 
		& Floor attendance (\%) & 0.897  & 0.819 &  0.0775** \\ 
		& &   &  &  (0.00478 ) \\ 
		& Number of Committees   &  2.77   &2.62 &  0.151**  \\ 
		& &   &  & (0.04649)  \\ 
		& Number of floor speeches   &  18.392 &16.649  &  1.74 \\ 
		& &   &  &  (1.266418) \\ 
		& Number of bills introduced  &  10.8  & 10.0 &  0.756 \\ 
		& &   &  & (0.6968)  \\ 
		& Vote with Party (\%)   &  0.842  & 0.759   &  0.0824** \\ 
		& &   &  &   (0.0053)\\ 
		\hline \\[-1.8ex] 
	\end{tabular} 
	\\
	\textit{Notes:} Standard errors are in parentheses. Short track corresponds to legislators that served from Legislatura LX to LXIII (Pre Electoral Reform). The number of observations is 2500. Significant at the 5\% level 
\end{table} 


\begin{table}[!htbp] \centering 
	\caption{Differences between \textbf{Type of Legislators}}
	\begin{tabular}{@{\extracolsep{4pt}} ccccc} 
		\\[-1.8ex]\hline 
		\hline \\[-1.8ex] 
		& Variable & PR & SSD & Difference of means  \\ 
		\hline \\[-1.8ex] 
		& Floor attendance (\%) &  0.822 &  0.844 &  -0.0222**   \\ 
		& &   &  &  (0.00516) \\ 
		& Number of Committees   & 2.649  &2.652  &  $-0.003$ \\ 
		& &   &  & (0.0605)  \\ 
		& Number of floor speeches   &  21.6 &15.8&  5.76**  \\ 
		& &   &  & (0.0017) \\ 
		& Number of bills introduced  & 12.0  & 8.97 &3.07** \\ 
		& &   &  &  (0.651) \\ 
		& Vote with Party (\%)   & 0.767 & 0.782    &   -0.0142**\\ 
		& &   &  & (0.0050)  \\ 
		\hline \\[-1.8ex] 
	\end{tabular} 
	\\
	\textit{Notes:} Standard errors are in parentheses. PR corresponds to proportional representation legislators. The number of observations is 2500. ** Significant at the 5\% level 
\end{table} 

\begin{table}[!htbp] \centering 
	\caption{Differences between period, within Type: \textbf{Proportional Representation}}
	\begin{tabular}{@{\extracolsep{4pt}} ccccc} 
		\\[-1.8ex]\hline 
		\hline \\[-1.8ex] 
		& Variable & Pre & Post & Difference of means  \\ 
		\hline \\[-1.8ex] 
		& Floor attendance (\%) & 0.803  &  0.891 & -0.0880**   \\ 
		& &   &  & (0.0085)  \\ 
		& Number of Committees   & 2.635   & 2.705  &   -0.07\\ 
		& &   &  &  (0.0774) \\ 
		& Number of floor speeches   &  22.4  &19.1& 3.33  \\ 
		& &   &  &  (2.14) \\ 
		& Number of bills introduced  & 12.2  &11.5  & 0.702 \\ 
		& &   &  & (1.30)  \\ 
		& Vote with Party (\%)   &  0.749 &  0.840   &  -0.0904** \\ 
		& &   &  &   \\ 
		\hline \\[-1.8ex] 
	\end{tabular} 
	\\
	\textit{Notes:} Standard errors are in parentheses. PR corresponds to proportional representation legislators. The number of observations is 2500. ** Significant at the 5\% level
\end{table} 

\begin{table}[!htbp] \centering 
	\caption{Differences between period, within Type of Legislator: \textbf{Single Seat District}}
	\begin{tabular}{@{\extracolsep{4pt}} ccccc} 
		\\[-1.8ex]\hline 
		\hline \\[-1.8ex] 
		& Variable & Pre & Post & Difference of means  \\ 
		\hline \\[-1.8ex] 
		& Floor attendance (\%) & 0.829  & 0.900 & -0.0707**  \\ 
		& &   &  &  (0.0055) \\ 
		& Number of Committees   &  2.61 &  2.82 & -0.206**   \\ 
		& &   &  &  (0.0576) \\ 
		& Number of floor speeches   &  15.0 &18.3 &  -3.35 \\ 
		& &   &  & (2.12) \\ 
		& Number of bills introduced  &  8.62  & 10.4  & -1.73**\\ 
		& &   &  &  (0.7676) \\ 
		& Vote with Party (\%)   &  0.766& 0.843   &   -0.0772**\\ 
		& &   &  &   \\ 
		\hline \\[-1.8ex] 
	\end{tabular} 
	\\
	\textit{Notes:} Standard errors are in parentheses. Pre corresponds to the period before the Electoral Reform; Post, after the Electoral Reform. The number of observations is 2500. Significant at the 5\% level 
\end{table} 



\end{document}
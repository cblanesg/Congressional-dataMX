\documentclass{article}
\usepackage[a4paper, total={7in, 10in}]{geometry}
\usepackage{subfig}
\usepackage{graphicx}
\usepackage{caption}
\usepackage{xcolor}
\usepackage{import}
\usepackage{hyperref}
\graphicspath{{../../data/02-outcomes/01-policy_positioning/01-text_analysis/wordfish/01_lda/descriptive/}{../../data/02-outcomes/01-policy_positioning/01-text_analysis/wordfish/04_ideal_figures/1-features/}{../../data/02-outcomes/01-policy_positioning/01-text_analysis/wordfish/04_ideal_figures/}}


\begin{document}
\section{Policy Estimation: Using text as data}

The second approach to study the policy positions of legislators in the Chamber considers their speeches in the parliamentary debate. Language and speech represent another political act that also help us to understand better their legislative behavior. As their are certain expectations and costs associated with speech related to reputation and party loyalty, the ideological content of their speeches can reflect strategic behavior with their party. To scale the policy position of the legislators I use Wordfish (Slapin and Proksh (2008)). This technique uses an un-supervised approach that not requires the use of reference texts. It analyzes the word frequencies of political texts and assume that the frequencies are generated by a Poisson process. The estimation requires three steps: defining policy dimensions, generating the word frequency dataset and running the algorithm. \\
\\
In the parliamentary debate, many policy issues are addressed that may be very different from each other, and more specifically, the vocabulary associated with that issue may be very particular. For this reason the words that will help us to discriminate between positions will certainly change according to the topic being discussed. To overcome this issue I perform a multidimensional analysis. First I identify the most salient topics in the parliamentary debate running a topic modeling analysis with Latent Dirichlet Allocation (LDA). Secondly, I retrieve separately the position of each legislator for each topic and subsequently, I calculate the a final policy position for each legislator by averaging all their positions observed in each topic. For the final analysis, I consider a policy position of each party and measure the distance between the member and the party for each Congress. This estimate will provide insights about the relationship between the member and the party, in other words, how distant a member is from the party agenda imposed by a party leader or the median speech of all the members. 

\subsubsection{LDA}

I classified parliamentary speeches into 20 topics estimated via an LDA model (Latent Dirichlet Allocation). In this model, each legislator speech is represented over latent word clusters or "topic". I labeled each of the 20 topics by analyzing the most salient words and documents with the highest likelihood for each topic. The following figures present the top 8 topics with their most salient words. 


\begin{minipage}{0.25\textwidth}\raggedleft
	\includegraphics[width=\linewidth]{wordcloud_topic2.png} \\
	\centering \scriptsize{Genero}
\end{minipage}
\begin{minipage}{0.25\textwidth}\raggedleft
	\includegraphics[width=\linewidth]{wordcloud_topic4.png} \\
	\centering \scriptsize{Salud}
\end{minipage}
\begin{minipage}{0.25\textwidth}\raggedleft
	\includegraphics[width=\linewidth]{wordcloud_topic8.png} \\
	\centering \scriptsize{Seguridad, Fuerzas Armadas}
\end{minipage}
\begin{minipage}{0.25\textwidth}\raggedleft
	\includegraphics[width=\linewidth]{wordcloud_topic20.png} \\
	\centering \scriptsize{Constitucional}
\end{minipage}
\begin{minipage}{0.25\textwidth}\raggedleft
	\includegraphics[width=\linewidth]{wordcloud_topic10.png} \\
	\centering \scriptsize{Derechos Nacionales }
\end{minipage}
\begin{minipage}{0.25\textwidth}\raggedleft
	\includegraphics[width=\linewidth]{wordcloud_topic13.png} \\
	\centering \scriptsize{Ciencia y Tecnología}
\end{minipage}
\begin{minipage}{0.25\textwidth}\raggedleft
	\includegraphics[width=\linewidth]{wordcloud_topic15.png} \\
	\centering \scriptsize{Energía}
\end{minipage}
\begin{minipage}{0.25\textwidth}\raggedleft
	\includegraphics[width=\linewidth]{wordcloud_topic16.png} \\
	\centering \scriptsize{Política Fiscal}
\end{minipage}

\begin{figure}
	\includegraphics[width=\linewidth]{topics_across_congress.png} \\
\end{figure}


\newpage
\subsubsection{Wordfish: Ideal Point Estimations by Topic}

For each congress I estimated the ideal points for each legislator for every all topics. Figure 1 shows the distribution of the ideal points of the members of Legislatura LXIV by party for the top 10 most spoken topics. 

\begin{minipage}{0.25\textwidth}\raggedleft
	\includegraphics[width=\linewidth]{genero60.png} \\
	\centering \scriptsize{Genero}
\end{minipage}
\begin{minipage}{0.25\textwidth}\raggedleft
	\includegraphics[width=\linewidth]{seguridad60.png} \\
	\centering \scriptsize{Seguridad, Fuerzas Armadas }
\end{minipage}
\begin{minipage}{0.25\textwidth}\raggedleft
	\includegraphics[width=\linewidth]{salud60.png} \\
	\centering \scriptsize{Salud}
\end{minipage}
\begin{minipage}{0.25\textwidth}\raggedleft
	\includegraphics[width=\linewidth]{educacion60.png} \\
	\centering \scriptsize{Educacion}
\end{minipage}
\begin{minipage}{0.25\textwidth}\raggedleft
	\includegraphics[width=\linewidth]{ciencia y tecnologia_60.png} \\
	\centering \scriptsize{Ciencia y Tecnologia}
\end{minipage}
\begin{minipage}{0.25\textwidth}\raggedleft
	\includegraphics[width=\linewidth]{consitutcional_60.png} \\
	\centering \scriptsize{Constitucional}
\end{minipage}
\begin{minipage}{0.25\textwidth}\raggedleft
	\includegraphics[width=\linewidth]{derechos nacionales_60.png} \\
	\centering \scriptsize{Derechos Nacionales}
\end{minipage}
\begin{minipage}{0.25\textwidth}\raggedleft
	\includegraphics[width=\linewidth]{fiscal, energia_60.png} \\
	\centering \scriptsize{Politica Fiscal}
\end{minipage}


\begin{figure}
	\centering
	\caption{}
	\label{} 
	\includegraphics[width=100mm,scale=0.5]{2-documents/genero_60.png} \\
\end{figure}

\begin{figure}
	\centering
	\caption{}
	\label{} 
	\includegraphics[width=150mm,scale=0.5]{legislatura64_density.png} \\
\end{figure}

\end{document}
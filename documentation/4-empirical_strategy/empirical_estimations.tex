\documentclass{article}
\usepackage[a4paper, total={7in, 10in}]{geometry}
\usepackage{subfig}
\usepackage{graphicx}
\usepackage{caption}
\usepackage{xcolor}
\usepackage{longtable}
\usepackage{import}
\usepackage{hyperref}

%\graphicspath{{../../data/02-deliberation/4-length_meetings/figures/}{../../data/02-deliberation/3-informativeness/2-descriptive/density_signals/}{../../data/04-reduce-form/2-regressions/}}

\begin{document}
\section{Data}
\section{Empirical Strategy and Analysis}

I implement a difference-in-differences estimator with the following specification:

\begin{equation}
Y_{i, t} = \alpha + \beta_1 T_t + \beta_2 E_i + \beta_3 T_t E_i + \mu_t + \gamma X_s+ \epsilon_{i, t},
\end{equation}

where $Y_{i,t}$ is an index for legislative effort, discipline with party in voting behavior or discipline with party in speech behavior for legislator i in period t (where \textit{t = \{Congress LX, LXI, LXII, LXIII, LXIV\}}). $\beta_1$ is the parameter that measures the effect of electoral institution (\textit{Proportional Representation (PR) or single member districts (SMD)}). The second parameter of interest is $\beta_2$ which represents a dummy that indicates if reelection incentives are present, $\beta_3$ measures the effect of the interaction of electoral institution and reelection incentives. $\mu_t$ is a congress fixed effect that controls for any time trends that confound the estimators. $X_i$ is a matrix of legislator characteristics (gender, level of education, previous political experience, experience in private sector, public administration, non-profit organizations or academic experience).\\
\\
The two main outcomes of interest are legislative effort and intra-party discipline. Specification (1) and (2) of Table 1 reports the estimates of the effect of electoral institution and reelection when the dependent variable is an index of legislative effort. The effect of the institution on effort indicates that members by PR exert a greater effort than SMD members, with and without reform. The sign of the coefficient is consistent controlling by party (1). Members in closed-list PR electoral systems place high value on  party unity because the elite party  can create career paths, therefore without the party any other future career will be impossible. By allocating a greater effort in legislative activities they signal their party that they are a more competent type. On the other side, SMD members not only value the relationship with their party, but also with their constituency to the extent that it can be used as a bargaining mechanism with the party. Parties benefit from popular SMD candidates since more votes the candidate get in the race, the more votes get party gains in PR. Thus, popular legislators have higher chances of being re-selected to run and therefore a higher bargaining power than PR members. We can observe a lesser effort by this type of legislators, since they are allocating their efforts in other types of activities that are more rewarding to their career, for example serving the constituency. Reelection incentives introduces two effects. On one side, more incentives to be accountable to the constituency and a higher discipline to the party since they control the re-nominations. The coefficient of reelection incentives (2), suggests that the party control offsets the accountability effect. SMD members now find it more rewarding to allocate more time to legislative activities. To determine whether the effects are concentrated in just one or two outcomes, Table 3 reports the effects on each effort measure. The effect of reelection is positive for three of the four measures and statistically significant for two of them. The point estimates suggests that legislators from both types of systems allocate effort in  different ways when there are no incentives for reelection, but when the electoral incentives are present, the differences across legislators decreases. \\
\\
To measure the effect of intra-party discipline we consider two types of behavior: voting and speech. Initially we would expect to see different results in the two measures because each behavior occurs at different stages of the legislative process and because of the  costs associated with each action are different. For instance, roll-call votes represent the ultimate dimension of the party agenda and voting against one's party would be the ultimate act of defiance. However, legislative speech also can deliver important insights about intra-party dynamics on how legislators shape policy-making. Specification (5) and (6) reports the estimates of the effect of electoral institution and reelection when the dependent variable is distance with party in parliamentary speech. The electoral institution seems to be playing an important role in explaining the differences in speech among legislators. Without reelection incentives SMD members deviate more from the party line than PR members. This behavior is consistent with the mechanisms previously stated, SMD can signal more disagreement than PR members because they have a better bargaining position.  Reelection incentives also introduces a change in behavior. The negative coefficient (consistent with party controls) suggests that the effect of the party is stronger than any accountability effect. SMD members will deviate less from the party line when the reelection incentives are present. In contrast to the speech, voting behavior presents less variation. Considering the party-centered environment in the chamber Intra-party cohesiveness in roll coll voting is extremely high, the neither the electoral institution is relevant to explain changes in voting behavior not the electoral reform. Members within the political bloc are not willing to deviate from the party line if it could jeopardize their political career. However, results show important insights about the intra-party dynamics in a party-centered environments. The deviation in speech and not in vote may be the result of a strategic decision of the legislator where he must trade-off between sticking with the party line or deviating to his true preferences. \\
\\ 
Previously it was stated that popular SMD member have more bargaining power than any other legislator because they can give more benefits to the party. To asses this mechanism, I include a variable that introduces the margin of victory in first-past-the-post districts to capture electoral safety. Legislators in "loose elections" have a margin of victory above average than other districts. These types of legislators, as being more popular, are more valuable to the party since they can position the party more strongly in the district for the next election. Table 2 presents the estimates with the same specification but including the electoral safety variable. *SMD loose election* is equal to one is the SMD member won the election with a margin of victory above the average of the other districts. The reference category are SMD members that won in a close election (margin of victory below the average of the other districts). As predicted, popular SMD members can exert lesser effort than SMD member in close elections. This suggest that non-popular members still need to signal the party that they are a competent type. On the other side, they deviate more than non-popular SMD members for the same reasons already mentioned. This results suggests that the type of institution the member are elected matters to analyze the relationship they have with their party. Although it was expected to see an increase accountability and a decentralization with the party with the reelection incentives, the estimates seem to present the opposite results. As long as the party continues to maintain control of nominations, at least with the SMD members, the party control effect will offset the accountability effect. 


\begin{table}[!htbp] \centering 
  \caption{} 
  \label{} 
\begin{tabular}{@{\extracolsep{5pt}}lcccccc} 
\\[-1.8ex]\hline 
\hline \\[-1.8ex] 
 & \multicolumn{6}{c}{\textit{Dependent variable:}} \\ 
\cline{2-7} 
\\[-1.8ex] & \multicolumn{2}{c}{legislative effort} & \multicolumn{2}{c}{distance w/party in votes} & \multicolumn{2}{c}{distance w/party in speech} \\ 
\\[-1.8ex] & (1) & (2) & (3) & (4) & (5) & (6)\\ 
\hline \\[-1.8ex] 
 SMD & $-$0.032 & $-$0.057$^{**}$ & 0.004 & 0.0001 & 0.071$^{***}$ & 0.082$^{***}$ \\ 
  & (0.024) & (0.024) & (0.007) & (0.007) & (0.021) & (0.021) \\ 
  & & & & & & \\ 
 reelection & 0.229$^{***}$ & 0.212$^{***}$ & $-$0.015 & $-$0.018 & $-$0.052 & $-$0.045 \\ 
  & (0.049) & (0.049) & (0.015) & (0.015) & (0.043) & (0.043) \\ 
  & & & & & & \\ 
   SMD x reelection & 0.011 & 0.081 & $-$0.010 & 0.002 & $-$0.025 & $-$0.057 \\ 
  & (0.054) & (0.053) & (0.016) & (0.016) & (0.048) & (0.047) \\ 
  & & & & & & \\ 
 main party & $-$0.286$^{***}$ &  & $-$0.050$^{***}$ &  & 0.124$^{***}$ &  \\ 
  & (0.042) &  & (0.013) &  & (0.036) &  \\ 
  & & & & & & \\ 
 female & 0.028 & 0.024 & 0.001 & 0.0002 & $-$0.032 & $-$0.030 \\ 
  & (0.022) & (0.023) & (0.007) & (0.007) & (0.019) & (0.019) \\ 
  & & & & & & \\ 
 education & 0.090$^{***}$ & 0.091$^{***}$ & $-$0.001 & $-$0.001 & $-$0.025$^{**}$ & $-$0.025$^{**}$ \\ 
  & (0.014) & (0.014) & (0.004) & (0.004) & (0.012) & (0.012) \\ 
  & & & & & & \\ 
 political experience & 0.026 & 0.028 & $-$0.001 & $-$0.0004 & 0.014 & 0.013 \\ 
  & (0.034) & (0.035) & (0.010) & (0.010) & (0.030) & (0.030) \\ 
  & & & & & & \\ 
 exp. private sector & $-$0.015 & $-$0.022 & 0.003 & 0.002 & 0.034$^{*}$ & 0.037$^{*}$ \\ 
  & (0.023) & (0.023) & (0.007) & (0.007) & (0.020) & (0.020) \\ 
  & & & & & & \\ 
 exp. public administration & $-$0.037 & $-$0.042$^{*}$ & $-$0.003 & $-$0.003 & 0.016 & 0.018 \\ 
  & (0.024) & (0.024) & (0.007) & (0.007) & (0.020) & (0.020) \\ 
  & & & & & & \\ 
 exp. non-profit org. & 0.045$^{**}$ & 0.051$^{**}$ & $-$0.013$^{*}$ & $-$0.012$^{*}$ & 0.043$^{**}$ & 0.039$^{**}$ \\ 
  & (0.023) & (0.023) & (0.007) & (0.007) & (0.020) & (0.020) \\ 
  & & & & & & \\ 
 academic experience & 0.111$^{***}$ & 0.112$^{***}$ & 0.006 & 0.006 & $-$0.091$^{***}$ & $-$0.092$^{***}$ \\ 
  & (0.024) & (0.024) & (0.007) & (0.007) & (0.020) & (0.020) \\ 
  & & & & & & \\ 
 Constant & 0.020 & $-$0.235$^{***}$ & 0.124$^{***}$ & 0.079$^{***}$ & 0.313$^{***}$ & 0.424$^{***}$ \\ 
  & (0.056) & (0.042) & (0.017) & (0.013) & (0.049) & (0.037) \\ 
  & & & & & & \\ 
\hline \\[-1.8ex] 
\textbf{Congress controls} & Yes & Yes & Yes & Yes & Yes & Yes \\ 
\textbf{Party controls} & Yes & No & Yes & No & Yes & No \\ 
Observations & 2,500 & 2,500 & 2,455 & 2,455 & 2,309 & 2,309 \\ 
Adjusted R$^{2}$ & 0.166 & 0.150 & 0.007 & 0.001 & 0.032 & 0.027 \\ 
\hline 
\hline \\[-1.8ex] 
\textit{Note:}  & \multicolumn{6}{r}{$^{*}$p$<$0.1; $^{**}$p$<$0.05; $^{***}$p$<$0.01} \\ 
 & \multicolumn{6}{r}{Standard Errors in parentheses} \\ 
\end{tabular} 
\end{table} 



\begin{table}[!htbp] \centering 
  \caption{} 
  \label{} 
\begin{tabular}{@{\extracolsep{5pt}}lccc} 
\\[-1.8ex]\hline 
\hline \\[-1.8ex] 
 & \multicolumn{3}{c}{\textit{Dependent variable:}} \\ 
\cline{2-4} 
\\[-1.8ex] & legislative effort & distance w/party in votes & distance w/party in speech \\ 
\\[-1.8ex] & (1) & (2) & (3)\\ 
\hline \\[-1.8ex] 
 SMD loose election & $-$0.054$^{*}$ & 0.007 & 0.083$^{***}$ \\ 
  & (0.031) & (0.009) & (0.027) \\ 
  & & & \\ 
 PR & 0.011 & $-$0.002 & $-$0.040$^{*}$ \\ 
  & (0.027) & (0.008) & (0.023) \\ 
  & & & \\ 
 reelection & 0.294$^{***}$ & $-$0.024 & $-$0.069 \\ 
  & (0.058) & (0.018) & (0.050) \\ 
  & & & \\ 
 main party & $-$0.291$^{***}$ & $-$0.050$^{***}$ & 0.125$^{***}$ \\ 
  & (0.042) & (0.013) & (0.036) \\ 
  & & & \\ 
 female & 0.030 & 0.001 & $-$0.033$^{*}$ \\ 
  & (0.022) & (0.007) & (0.019) \\ 
  & & & \\ 
 education & 0.088$^{***}$ & $-$0.001 & $-$0.023$^{*}$ \\ 
  & (0.014) & (0.004) & (0.012) \\ 
  & & & \\ 
 political experience & 0.028 & $-$0.001 & 0.013 \\ 
  & (0.034) & (0.010) & (0.030) \\ 
  & & & \\ 
 exp. private sector & $-$0.015 & 0.003 & 0.035$^{*}$ \\ 
  & (0.023) & (0.007) & (0.020) \\ 
  & & & \\ 
 exp. public administration & $-$0.035 & $-$0.003 & 0.012 \\ 
  & (0.024) & (0.007) & (0.020) \\ 
  & & & \\ 
 exp. non-profit org. & 0.045$^{*}$ & $-$0.013$^{*}$ & 0.044$^{**}$ \\ 
  & (0.023) & (0.007) & (0.020) \\ 
  & & & \\ 
 academic experience & 0.109$^{***}$ & 0.006 & $-$0.089$^{***}$ \\ 
  & (0.024) & (0.007) & (0.020) \\ 
  & & & \\ 
 SMD loose election x reelection & $-$0.080 & $-$0.004 & $-$0.035 \\ 
  & (0.068) & (0.021) & (0.059) \\ 
  & & & \\ 
 PR x reelection & $-$0.070 & 0.009 & 0.021 \\ 
  & (0.066) & (0.020) & (0.057) \\ 
  & & & \\ 
 Constant & 0.019 & 0.125$^{***}$ & 0.347$^{***}$ \\ 
  & (0.058) & (0.018) & (0.051) \\ 
  & & & \\ 
\hline \\[-1.8ex] 
\textbf{Congress controls} & Yes & Yes & Yes \\ 
\textbf{Party controls} & Yes & Yes & Yes \\ 
Observations & 2,500 & 2,455 & 2,309 \\ 
Adjusted R$^{2}$ & 0.168 & 0.006 & 0.035 \\ 
\hline 
\hline \\[-1.8ex] 
\textit{Note:}  & \multicolumn{3}{r}{$^{*}$p$<$0.1; $^{**}$p$<$0.05; $^{***}$p$<$0.01} \\ 
 & \multicolumn{3}{r}{Standard Errors in parentheses} \\ 
\end{tabular} 
\end{table} 


\begin{table}[!htbp] \centering 
  \caption{Effect of Electoral Reform and Electoral Institution over Legislative Effort: Productivity Measures} 
  \label{} 
\begin{tabular}{@{\extracolsep{5pt}}lcccc} 
\\[-1.8ex]\hline 
\hline \\[-1.8ex] 
 & \multicolumn{4}{c}{\textit{Dependent variable:}} \\ 
\cline{2-5} 
\\[-1.8ex] & Floor Attendance & Committee Participation & Floor Speech & Bills introduced \\ 
\\[-1.8ex] & (1) & (2) & (3) & (4)\\ 
\hline \\[-1.8ex] 
 SMD & 0.017$^{**}$ & 0.041 & $-$0.285 & $-$1.347$^{**}$ \\ 
  & (0.009) & (0.056) & (1.572) & (0.667) \\ 
  & & & & \\ 
 reelection & 0.126$^{***}$ & $-$0.078 & 4.047 & 2.919$^{**}$ \\ 
  & (0.018) & (0.114) & (3.223) & (1.368) \\ 
  & & & & \\ 
   SMD x reelection & $-$0.014 & 0.027 & $-$6.477$^{*}$ & $-$0.778 \\ 
  & (0.019) & (0.121) & (3.436) & (1.458) \\ 
  & & & & \\ 
 female & 0.002 & 0.077 & $-$1.818 & 0.868 \\ 
  & (0.008) & (0.049) & (1.398) & (0.593) \\ 
  & & & & \\ 
 education & 0.032$^{***}$ & 0.091$^{***}$ & 2.190$^{**}$ & 0.613$^{*}$ \\ 
  & (0.005) & (0.030) & (0.862) & (0.366) \\ 
  & & & & \\ 
 political experience & 0.040$^{***}$ & 0.033 & $-$2.994 & $-$0.169 \\ 
  & (0.012) & (0.076) & (2.166) & (0.919) \\ 
  & & & & \\ 
 exp. private sector & 0.016$^{**}$ & 0.082 & $-$4.325$^{***}$ & $-$1.343$^{**}$ \\ 
  & (0.008) & (0.051) & (1.451) & (0.616) \\ 
  & & & & \\ 
 exp. public administration & 0.002 & $-$0.122$^{**}$ & 0.023 & 0.211 \\ 
  & (0.008) & (0.052) & (1.482) & (0.629) \\ 
  & & & & \\ 
 exp. non-profit org. & 0.034$^{***}$ & 0.035 & $-$0.418 & 0.155 \\ 
  & (0.008) & (0.051) & (1.455) & (0.618) \\ 
  & & & & \\ 
 academic experience & 0.023$^{***}$ & $-$0.006 & 7.009$^{***}$ & 2.635$^{***}$ \\ 
  & (0.008) & (0.052) & (1.476) & (0.626) \\ 
  & & & & \\ 
 Constant & 0.620$^{***}$ & 2.655$^{***}$ & 20.328$^{***}$ & 13.827$^{***}$ \\ 
  & (0.035) & (0.222) & (6.275) & (2.663) \\ 
  & & & & \\ 
\hline \\[-1.8ex] 
\textbf{Congress controls} & Yes & Yes & Yes & Yes \\ 
\textbf{Party controls} & Yes & Yes & Yes & Yes \\ 
Observations & 2,500 & 2,500 & 2,500 & 2,500 \\ 
Adjusted R$^{2}$ & 0.148 & 0.391 & 0.128 & 0.118 \\ 
\hline 
\hline \\[-1.8ex] 
\textit{Note:}  & \multicolumn{4}{r}{$^{*}$p$<$0.1; $^{**}$p$<$0.05; $^{***}$p$<$0.01} \\ 
 & \multicolumn{4}{r}{Standard Errors in parentheses} \\ 
\end{tabular} 
\end{table} 


\end{document}
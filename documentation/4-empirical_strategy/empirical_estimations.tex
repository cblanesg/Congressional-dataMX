\documentclass{article}
\usepackage[a4paper, total={6in, 7in}]{geometry}
\usepackage{subfig}
\usepackage{graphicx}
\usepackage{caption}
\usepackage{xcolor}
\usepackage{longtable}
\usepackage{import}
\usepackage{hyperref}

\graphicspath{{../../data/02-outcomes/01-policy_positioning/00-roll_call/02-graphs/}{../../data/02-outcomes/01-policy_positioning/01-text_analysis/wordfish/01_lda/descriptive/}{../../data/02-outcomes/01-policy_positioning/01-text_analysis/wordfish/04_ideal_figures/}{../../data/02-outcomes/01-policy_positioning/01-text_analysis/wordfish/04_ideal_figures/1-features/}}

\begin{document}
	
\section{Data and Measures of Legislative Behavior}

In order to study the legislative behavior of legislators in the Chamber of Deputies, I collected a dataset containing different variables that reflect legislative effort, voting and speech behavior. 

This section provides an overview of the data and how it was collected. All the raw data was obtained from the Congressional webpage \url{http://www.diputados.gob.mx}. So far, there was no consolidated  database with all the information available on the legislators. I applied web-scraping techniques to collect all the available data,  process it and obtain a structured base in order to study legislative behavior. The final data set contains  information on individual behavior and characteristics of the legislator aggregated at the congressional period for all congress for congresses in session between 2012 and 2020. The sample includes 2500 legislators (excluding all substitutes) in 60th Legislature (2006-09), the 61nd (2009-12), and the 62th (2012-15), 63th Legislature (2012-18)and 64th Legislature (2018-21). \\
\\
Among the collected information, I obtained demographics of the legislators such as gender and background experience. Each legislator reports his or her previous experience in any sector along with his or her education. To aggregate all this information into the same categories, I hand-coded X categories regarding background experience and levels of education. Table 1 presents each category with a couple of examples of the self-reported experience. 

\begin{table}[h!] \centering 
	\caption{Correlations among measures of legislative effort}
	\begin{tabular}{@{\extracolsep{4pt}} ll} 
		\hline 
		\hline \\[-1.8ex] 
		    &   Experience   \\ 
		\hline \\[-1.8ex] 
		
		\textbf{Background Experience} &     \\ 
	\hline \\[-1.8ex] 
		Political Experience &  \textit{Gobernador, Presidente Municipal, etc}   \\ 
		Private Sector &   \textit{Director Ejecutivo en empresa X} \\ 
		Public Administration &   \textit{Directora de Proyectos Especiales del Ejecutivo, etc.} \\ 
		Non-profit organizations &  \textit{organismo internacional}  \\ 
		Academy &   \textit{Profesor/a en escuela X} \\ 
	\hline \\[-1.8ex] 
		\textbf{Education Level} &     \\ 
		\hline \\[-1.8ex] 
		Basica &  \textit{Primaria, Secundaria}  \\ 
		Media-Superior &  \textit{Carrera Técnica, Bachillerato, Nivel Medio Superior}  \\ 
		Superior &   \textit{Licenciatura, Maestria, Doctorado, Especialidad} \\ 
		\hline
		\hline 
	\end{tabular} 
	\caption*{\textit{Note: Correlations computed on raw data observed on a congressional period. }}
\end{table} 


\subsection{Effort Measure}
The first outcome of interest is the effort behavior the legislators exert in legislative activities. Therefore, I consider the full range of variables proposed by Dal Bo and Rossi, which includes: \textbf{floor attendance} (as percentage of legislative floor sessions), the number of \textbf{legislative bills} in which the legislator participated, the \textbf{number of committees} to which the legislator is a member, and finally the number of \textbf{floor speeches} given by the legislator on a legislative topic. \\


\begin{table}[h!] \centering 
	\caption{Correlations among measures of legislative effort}
	\begin{tabular}{@{\extracolsep{4pt}} cccccc} 
		\hline 
		\hline \\[-1.8ex] 
		& Number   &   Floor  & Bills  & Floor \\ 
		&  Committees  &    Speeches &  Introduced &  Attendance\\ 
		\hline \\[-1.8ex] 
		Number Committees &  1  &    &   &\\ 
		Floor Speeches  	   &  0.07  &   1 &   &\\ 
		Bills Introduced 	  &  -0.06  &  0.46  & 1  &\\ 
		Floor Attendance		 &  0.05  &  0.05  &  0.07  &1\\ 
		\hline 
	\end{tabular} 
	\caption*{\textit{Note: Correlations computed on raw data observed on a congressional period. }}
\end{table} 

In order to analyze \textit{legislative effort} as a broader definition, I construct and index of legislative effort that aggregated the four measures described above. Following the same steps as DalBo, the index is the equally weighted average of the  z-scores of its component. The z-scores are levels standardized using the mean and standard deviation for the six-year legislators. In all cases, higher-effort measures have higher z-scores.


\subsection{Voting  Behavior}

To study how legislators vote on the floor, I collected data on roll-call votes, again, from the Congressional official website. In total, I was able to retrieve over XXX  individual roll-call votes of all legislators across the five congressional periods. 

To study ideal positioning of legislators in the Chamber, from this data I scale legislators ideological points, using DW-NOMINATE algorithm. The algorithm assumes probabilistic voting based on a spatial utility function. The parameters of the utility function and the spatial coordinates of the legislators and the votes are estimated on the basis of observed voting behavior.

As with any scaling method, the polarities of the estimated coordinates in W-NOMINATE are arbitrary. It is required the specification of a legislator to be positive in each dimension. To orient the estimated results and therefore be consistent across terms, I obtained the Poole’s Optimal Classification scores from the roll call votes and obtain the most conservative legislator of the party \textit{”Partido Accion Nacional”}, which is consider the most conservative party. This resulting legislator will be the specification for the W-NOMINATE scaling and therefore set the direction of the scores. Having conservatives on the right and liberals on the left, so ‘positive’ in this case generally means conservative. For each congress I estimated the ideal points of each legislator. I use the \textbf{wnominate} package in R to implement the scaling, and extracted the first dimension of the score as the measure of ideological positioning which will be used to obtain a measure of party discipline. Figures 2 to 8 show the ideological positioning of each legislator for each congress. All legislatures show, except LXIII, a consistent positioning of the ideological parties: PAN is always far-right,  more liberal parties are situated in the left (PRD, Morena) and PRI at the center. Legislature LXIV is the congress with more polarization in voting patterns, this could be associated to the fact that the three-plus party system came crashing down in the critical election of 2018, with Morena winning a  majority in Congress. 

\begin{figure}[!h]
	\centering
	\begin{minipage}{.5\textwidth}
		\centering
		\caption{Ideal Point Estimates \textit{Legislatura LX}}
		\includegraphics[width=1\linewidth]{dw_nominateLX.png}
		\label{fig:test1}
	\end{minipage}%
	\begin{minipage}{.5\textwidth}
		\centering
		\caption{\textit{Legislatura LX}}
		\includegraphics[width=1\linewidth]{dw_nominate_densityLX.png}
		\label{fig:test2}
	\end{minipage}
\end{figure}

\begin{figure}[!h]
	\centering
	\begin{minipage}{.5\textwidth}
		\centering
		\caption{\textit{Legislatura LXI}}
		\includegraphics[width=1\linewidth]{dw_nominateLXI.png}
		\label{fig:test1}
	\end{minipage}%
	\begin{minipage}{.5\textwidth}
		\centering
		\caption{\textit{Legislatura LXI}}
		\includegraphics[width=1\linewidth]{dw_nominate_densityLXI.png}
		\label{fig:test2}
	\end{minipage}
\end{figure} 

\begin{figure}[h!]
	\centering
	\begin{minipage}{.5\textwidth}
		\centering
		\caption{\textit{Legislatura LXII}}
		\includegraphics[width=1\linewidth]{dw_nominateLXII.png}
		\label{fig:test1}
	\end{minipage}%
	\begin{minipage}{.5\textwidth}
		\centering
		\caption{Ideal Point Density \textit{Legislatura LXII}}
		\includegraphics[width=1\linewidth]{dw_nominate_densityLXII.png}
		\label{fig:test1}
	\end{minipage}%
\end{figure}

\begin{figure}[!h]
	\centering
	\begin{minipage}{.5\textwidth}
		\centering
		\caption{\textit{Legislatura LXIII}}
		\includegraphics[width=1\linewidth]{dw_nominateLXIII.png}
		\label{fig:test1}
	\end{minipage}%
	\begin{minipage}{.5\textwidth}
		\centering
		\caption{\textit{Legislatura LXIII}}
		\includegraphics[width=1\linewidth]{dw_nominate_densityLXIII.png}
		\label{fig:test2}
	\end{minipage}
\end{figure}

\begin{figure}[!h]
	\centering
	\begin{minipage}{.5\textwidth}
		\centering
		\caption{\textit{Legislatura LXIV}}
		\includegraphics[width=1\linewidth]{dw_nominateLXIV.png}
		\label{fig:test1}
	\end{minipage}%
	\begin{minipage}{.5\textwidth}
		\centering
		\caption{\textit{Legislatura LXIV}}
		\includegraphics[width=1\linewidth]{dw_nominate_densityLXIV.png}
		\label{fig:test2}
	\end{minipage}
\end{figure}  

To analyze party discipline in voting behavior patterns, I use the ideal point estimations to estimate the distance between each member and their party. I use the median legislator of each party as a proxy of the party's ideal point. Finally, I define party discipline as the difference between each member's ideal point with the party ideal point. 


\newpage
\subsection{Speech  Behavior}

The third and final measure I'm interested in, is the speech behavior.  Speeches were digitized by the stenographic service (and scraped from \url{http://cronica.diputados.gob.mx/Intervenciones/index.html}).  Language and speech represent another political act that also help us to understand better their legislative behavior. As there are certain expectations and costs associated with speech related to party loyalty, this source of information can provide insights about the relationship between members and party. 

I preprocessed the documents following standard practice in text analysis methods. First I removed punctuation, capitalization, numbers and stop-words. Second, I create bigrams (two word phrases) and unigrams (one word phrases) and apply an stemming algorithm to convert terms into the same linguistic root. With all these steps there are XXX terms in the vocabulary of the speeches of all congressional periods. After all the basic processing steps I removed frequent and un-frequent terms from the corpus, defined as those that appear in less than 0.1\% of all documents (each intervention is considered as a document). With this procedure, the final corpus contains XXX terms that will be used to scale the preferences of the legislators. 

To quantify the legislator ideal points with text data, I scale the policy position of the legislators using  Wordfish (Slapin and Proksh (2008)). This technique uses an un-supervised approach that not requires the use of reference texts. It analyzes the word frequencies of political texts and assume that the frequencies are generated by a Poisson process. The estimation requires three steps: defining policy dimensions, generating the word frequency data set and running the algorithm. 

In the parliamentary debate there is no restriction on the

Like the scaling technique used for voting patterns, this approach assumes the relative word usage of parties provides information about their placement in a policy space. The technique can be understood as follows: A speech is represented as a vector of word counts or occurrences and individual words are assumed to be distributed at random. The word frequencies are generated by a Poisson process (number of times party $i$ mentions word $q$ in congress $t$ is drawn from a Poisson distribution). Additionally, It assumes that the parameter associated to the Poisson distribution $\lambda$ is a function of other parameters. More specifically, 
\begin{equation}
\lambda_{itq} = exp(\alpha_{it} + \psi_q + \beta_q * \omega_{iq})
\end{equation}

where $\alpha$ is a set of party-election year fixed effects, $\psi$ is a set of word fixed effects,  $\beta$ is an estimate of a word specific weight capturing the importance of word $q$ in discriminating between party positions and $\omega$ is the estimate of party i's position in election year t. Word fixed effects controls for some words that can be associated to words that are more associated with some parties. Party fixed effects control for the possibility that some parties could have spoken more in one congress than other. 

The parliamentary debate covers all policy issues, if the vocabulary associated with each issue is very particular, the parameter $\beta$  that captures the importance of the word in discriminating along policy positions will be different for each topic. Therefore, I identify the most salient topics in the parliamentary debate running a topic modeling analysis with Latent Dirichlet Allocation (LDA). Secondly, I retrieve separately the position of each legislator for each topic and subsequently, I calculate the a final policy position for each legislator by averaging all their positions observed in each topic. 

\subsubsection{Topic Modeling}

I classified parliamentary speeches into 20 topics estimated via an LDA model (Latent Dirichlet Allocation). In this model, each legislator speech is represented over latent word clusters or ”topic”. I labeled each of the 20 topics by analyzing the most salient words and documents with the highest likelihood for each topic. The following figures present the top 8 topics with their most salient words. 

For the final analysis, I consider a policy position of each party and measure the distance between the member and the party for each Congress. This estimate will provide insights about the relationship between the member and the party, in other words, how distant a member is from the party agenda imposed by a party leader or the median speech of all the members. For each topic, I present a wordcloud representation, with larger words being more highly associated with the topic (figure X to X). Terms such as "mujer" , "discriminacion" , "derecho",  "violencia" were assigned to \textit{Gender Topics}, Terms sucha as "salud", "enfermedad", "tratamiento", "prevencion" were assigned to \textit{Health topics}. \\
\\
\begin{minipage}{0.25\textwidth}\raggedleft
	\includegraphics[width=\linewidth]{wordcloud_topic2.png} \\
	\centering \scriptsize{Genero}
\end{minipage}
\begin{minipage}{0.25\textwidth}\raggedleft
	\includegraphics[width=\linewidth]{wordcloud_topic4.png} \\
	\centering \scriptsize{Salud}
\end{minipage}
\begin{minipage}{0.25\textwidth}\raggedleft
	\includegraphics[width=\linewidth]{wordcloud_topic8.png} \\
	\centering \scriptsize{Seguridad, Fuerzas Armadas}
\end{minipage}
\begin{minipage}{0.25\textwidth}\raggedleft
	\includegraphics[width=\linewidth]{wordcloud_topic20.png} \\
	\centering \scriptsize{Constitucional}
\end{minipage}
\begin{minipage}{0.25\textwidth}\raggedleft
	\includegraphics[width=\linewidth]{wordcloud_topic10.png} \\
	\centering \scriptsize{Derechos Nacionales }
\end{minipage}
\begin{minipage}{0.25\textwidth}\raggedleft
	\includegraphics[width=\linewidth]{wordcloud_topic13.png} \\
	\centering \scriptsize{Ciencia y Tecnología}
\end{minipage}
\begin{minipage}{0.25\textwidth}\raggedleft
	\includegraphics[width=\linewidth]{wordcloud_topic15.png} \\
	\centering \scriptsize{Energía}
\end{minipage}
\begin{minipage}{0.25\textwidth}\raggedleft
	\includegraphics[width=\linewidth]{wordcloud_topic16.png} \\
	\centering \scriptsize{Política Fiscal}
\end{minipage}


Figure 11 shows the number of speeches of each topic across congresses. Although some topics were spoken with a higher proportion (e.g: responsabilidad administrativa, Legislature LXI), all topics are covered in all Legislatures. 
\begin{figure}[!h]
		\caption{Topics Across Congress}
		\centering
	\includegraphics[width=1\linewidth]{topics_across_congress.png} \\
\end{figure}

\subsubsection{Scaling Political Texts: Wordfish}

For each congress I estimated the ideal points for each legislator for all topics. I use the \textbf{wordfish} package in R to implement the scaling. First, I estimate the parameter $\beta_q$  that captures the weight of every word that discriminates between policy positions.  Figure X shows the weight assigned to each word in \textit{Gender Topics}. Next, I obtained the ideal point estimation for every legislator averaging across all points in the policy issues. Figure X plots the ideal point estimations in Legislature LX of  \textit{Gender Topics}. \\
\\
\begin{minipage}{0.5\textwidth}\raggedleft
	\includegraphics[width=\linewidth]{genero_60.png} \\
	\centering \scriptsize{Word Weights: Gender Topics}
\end{minipage}
\begin{minipage}{0.5\textwidth}\raggedleft
	\includegraphics[width=\linewidth]{2-documents/genero_60.png} \\
	\centering \scriptsize{Ideal Points: Gender Topics}
\end{minipage}

To analyze party discipline relevant in speech discourses, I use the median legislator of each party and estimate the difference between ideal points of the members and the party. 

\section{Empirical Strategy and Analysis}

I implement a difference-in-differences estimator with the following specification:

\begin{equation}
Y_{i, t} = \alpha + \beta_1 T_t + \beta_2 E_i + \beta_3 T_t E_i + \mu_t + \gamma X_s+ \epsilon_{i, t},
\end{equation}

where $Y_{i,t}$ is an index for legislative effort, discipline with party in voting behavior or discipline with party in speech behavior for legislator i in period t (where \textit{t = \{Legislature LX, LXI, LXII, LXIII, LXIV\}}). $\beta_1$ is the parameter that measures the effect of electoral institution (\textit{Proportional Representation (PR) or single member districts (SMD)}). The second parameter of interest is $\beta_2$ which represents a dummy that indicates if reelection incentives are present, $\beta_3$ measures the effect of the interaction of electoral institution and reelection incentives. $\mu_t$ is a congress fixed effect and $X_i$ is a matrix of legislator characteristics (gender, level of education, previous political experience, experience in private sector, public administration, non-profit organizations or academic experience).\\
\\
The two main outcomes of interest are legislative effort and intra-party discipline. Columns (1) and (2) of Table 1 reports the estimates of the effect of electoral institution and reelection when the dependent variable is an index of legislative effort. The effect of the institution on effort indicates that members by PR exert a greater effort than SMD members when there are no reelection incentives. The sign of the coefficient is consistent controlling for party size (1 if the legislator is in one of the parties with the largest number of members). The underlying mechanism can be explained by how members place value across party or other interests such as their constituency of own preferences. Members in closed-list PR electoral systems place high value on  party unity because the elite party has a big leverage over career paths, therefore without the party any other future career in the legislative arena will be impossible. By allocating a greater effort in legislative activities they signal their party leaders that they are a more competent type. On the other side, SMD members not only value the relationship with their party, but also with their constituency to the extent that it can be used as a bargaining mechanism with the party. Parties benefit from popular SMD candidates since more votes the candidate get in the race, the more votes get party gains in PR. Thus, popular legislators have higher chances of being re-selected to run and therefore a higher bargaining power than PR members. We can observe a lesser effort by this type of legislators, since they are allocating their efforts in other types of activities that are more rewarding to their career, for example serving the constituency. Reelection incentives introduces two effects. On one side, more incentives to be accountable to the constituency and a higher discipline to the party since they control the re-nominations. The coefficient of reelection incentives (2), suggests that the party control offsets the accountability effect. SMD members now find it more rewarding to allocate more time to legislative activities. To determine whether the effects are concentrated in just one or two outcomes, Table 3 reports the effects on each effort measure. The effect of reelection is positive for three of the four measures and statistically significant for two of them. The point estimates suggests that legislators from both types of systems allocate effort in  different ways when there are no incentives for reelection, but when the electoral incentives are present, the differences across legislators decreases. \\
\\
To measure the effect of intra-party discipline we consider two types of behavior: voting and speech. Initially we would expect to see different results in the two measures because each behavior occurs at different stages of the legislative process and because of the costs associated with each action are different. For instance, roll-call votes represent the ultimate dimension of the party agenda and voting against one's party would be the ultimate act of defiance. However, legislative speech also can deliver important insights about intra-party dynamics on how legislators shape policy-making. columns (5) and (6) reports the estimates of the effect of electoral institution and reelection when the dependent variable is distance with party in parliamentary speech. The electoral institution seems to be playing an important role in explaining the differences in speech among legislators. Without reelection incentives SMD members deviate more from the party line than PR members. This behavior is consistent with the mechanisms previously stated, SMD can signal more disagreement than PR members because they have a better bargaining position. Reelection incentives also introduces a change in behavior. The negative coefficient (consistent with party controls) suggests that the effect of the party is stronger than any accountability effect. SMD members will deviate less from the party line when the reelection incentives are present. In contrast to the speech, voting behavior presents less variation. Considering the party-centered environment, intra-party cohesiveness in roll coll voting is extremely high, and neither the electoral institution is relevant to explain changes in voting behavior not the electoral reform. Members within the political bloc are not willing to deviate from the party line if it could jeopardize their political career. However, results show important insights about the intra-party dynamics in a party-centered environments. The deviation in speech and not in vote may be the result of a strategic decision of the legislator where he must trade-off between sticking with the party line or deviating to his true preferences or to the constituency. \\
\\ 
Previously it was stated that popular SMD member have more bargaining power than any other legislator because they can give more benefits to the party. To asses this mechanism, I include a variable that introduces the margin of victory in first-past-the-post districts to capture electoral safety. Legislators in "loose elections" have a margin of victory above average than other districts. These types of legislators, as being more popular, are more valuable to the party since they can position the party more strongly in the district for the next election. Table 2 presents the estimates with the same specification but including the electoral safety variable. *SMD loose election* is equal to one is the SMD member won the election with a margin of victory above the average of the other districts. The reference category are SMD members that won in a close election (margin of victory below the average of the other districts). As predicted, popular SMD members can exert lesser effort than SMD member in close elections. This suggest that non-popular members still need to signal the party that they are a competent type. On the other side, they deviate more than non-popular SMD members for the same reasons already mentioned. This results suggests that the type of institution the member are elected matters to analyze the relationship they have with their party. Although it was expected to see an increase accountability and a decentralization with the party with the reelection incentives, the estimates seem to present the opposite results. As long as the party continues to maintain control of nominations, at least with the SMD members, the party control effect will offset the accountability effect. 

\begin{table}[!htbp] \centering 
  \caption{Effect of Electoral Reform and Electoral Institution over Legislative Effort: Productivity Measure} 
  \label{} 
\begin{tabular}{@{\extracolsep{5pt}}lcccccc} 
\\[-1.8ex]\hline 
\hline \\[-1.8ex] 
 & \multicolumn{6}{c}{\textit{Dependent variable:}} \\ 
\cline{2-7} 
\\[-1.8ex] & \multicolumn{2}{c}{legislative effort} & \multicolumn{2}{c}{distance w/party in votes} & \multicolumn{2}{c}{distance w/party in speech} \\ 
\\[-1.8ex] & (1) & (2) & (3) & (4) & (5) & (6)\\ 
\hline \\[-1.8ex] 
 SMD & $-$0.009 & $-$0.057$^{**}$ & 0.004 & 0.0001 & 0.059$^{***}$ & 0.082$^{***}$ \\ 
  & (0.024) & (0.024) & (0.007) & (0.007) & (0.021) & (0.021) \\ 
  & & & & & & \\ 
 reelection & 0.176$^{***}$ & 0.212$^{***}$ & $-$0.022 & $-$0.018 & $-$0.022 & $-$0.045 \\ 
  & (0.048) & (0.049) & (0.015) & (0.015) & (0.043) & (0.043) \\ 
  & & & & & & \\ 
 party size & $-$0.264$^{***}$ &  & $-$0.024$^{***}$ &  & 0.125$^{***}$ &  \\ 
  & (0.026) &  & (0.008) &  & (0.023) &  \\ 
  & & & & & & \\ 
 female & 0.028 & 0.024 & 0.001 & 0.0002 & $-$0.032$^{*}$ & $-$0.030 \\ 
  & (0.022) & (0.023) & (0.007) & (0.007) & (0.019) & (0.019) \\ 
  & & & & & & \\ 
 education & 0.080$^{***}$ & 0.091$^{***}$ & $-$0.002 & $-$0.001 & $-$0.019 & $-$0.025$^{**}$ \\ 
  & (0.014) & (0.014) & (0.004) & (0.004) & (0.012) & (0.012) \\ 
  & & & & & & \\ 
 political experience & 0.037 & 0.028 & 0.001 & $-$0.0004 & 0.007 & 0.013 \\ 
  & (0.034) & (0.035) & (0.010) & (0.010) & (0.030) & (0.030) \\ 
  & & & & & & \\ 
 exp. private sector & $-$0.012 & $-$0.022 & 0.003 & 0.002 & 0.032$^{*}$ & 0.037$^{*}$ \\ 
  & (0.022) & (0.023) & (0.007) & (0.007) & (0.019) & (0.020) \\ 
  & & & & & & \\ 
 exp. public administration & $-$0.031 & $-$0.042$^{*}$ & $-$0.002 & $-$0.003 & 0.012 & 0.018 \\ 
  & (0.023) & (0.024) & (0.007) & (0.007) & (0.020) & (0.020) \\ 
  & & & & & & \\ 
 exp. non-profit org. & 0.050$^{**}$ & 0.051$^{**}$ & $-$0.013$^{*}$ & $-$0.012$^{*}$ & 0.042$^{**}$ & 0.039$^{**}$ \\ 
  & (0.023) & (0.023) & (0.007) & (0.007) & (0.020) & (0.020) \\ 
  & & & & & & \\ 
 academic experience & 0.108$^{***}$ & 0.112$^{***}$ & 0.006 & 0.006 & $-$0.090$^{***}$ & $-$0.092$^{***}$ \\ 
  & (0.023) & (0.024) & (0.007) & (0.007) & (0.020) & (0.020) \\ 
  & & & & & & \\ 
 SMD x reelection & 0.038 & 0.081 & $-$0.001 & 0.002 & $-$0.040 & $-$0.057 \\ 
  & (0.053) & (0.053) & (0.016) & (0.016) & (0.047) & (0.047) \\ 
  & & & & & & \\ 
 Constant & $-$0.021 & $-$0.235$^{***}$ & 0.098$^{***}$ & 0.079$^{***}$ & 0.323$^{***}$ & 0.424$^{***}$ \\ 
  & (0.047) & (0.042) & (0.014) & (0.013) & (0.041) & (0.037) \\ 
  & & & & & & \\ 
\hline \\[-1.8ex] 
\textbf{Congress controls} & Yes & Yes & Yes & Yes & Yes & Yes \\ 
Observations & 2,499 & 2,500 & 2,454 & 2,455 & 2,308 & 2,309 \\ 
Adjusted R$^{2}$ & 0.183 & 0.150 & 0.004 & 0.001 & 0.039 & 0.027 \\ 
\hline 
\hline \\[-1.8ex] 
\textit{Note:}  & \multicolumn{6}{r}{$^{*}$p$<$0.1; $^{**}$p$<$0.05; $^{***}$p$<$0.01} \\ 
 & \multicolumn{6}{r}{Standard Errors in parentheses} \\ 
\end{tabular} 
\end{table} 


\begin{table}[!htbp] \centering 
  \caption{Effect of Electoral Reform and Electoral Institution over Legislative Effort and Party Discipline: Popular Candidates} 
  \label{} 
\begin{tabular}{@{\extracolsep{5pt}}lccc} 
\\[-1.8ex]\hline 
\hline \\[-1.8ex] 
 & \multicolumn{3}{c}{\textit{Dependent variable:}} \\ 
\cline{2-4} 
\\[-1.8ex] & legislative effort & distance w/party in votes & distance w/party in speech \\ 
\\[-1.8ex] & (1) & (2) & (3)\\ 
\hline \\[-1.8ex] 
 SMD loose election & $-$0.041 & 0.008 & 0.076$^{***}$ \\ 
  & (0.031) & (0.009) & (0.026) \\ 
  & & & \\ 
 PR & $-$0.006 & $-$0.001 & $-$0.032 \\ 
  & (0.027) & (0.008) & (0.023) \\ 
  & & & \\ 
 reelection & 0.230$^{***}$ & $-$0.027 & $-$0.038 \\ 
  & (0.058) & (0.018) & (0.050) \\ 
  & & & \\ 
 size party & $-$0.260$^{***}$ & $-$0.024$^{***}$ & 0.121$^{***}$ \\ 
  & (0.027) & (0.008) & (0.023) \\ 
  & & & \\ 
 female & 0.029 & 0.0005 & $-$0.033$^{*}$ \\ 
  & (0.022) & (0.007) & (0.019) \\ 
  & & & \\ 
 education & 0.079$^{***}$ & $-$0.002 & $-$0.019 \\ 
  & (0.014) & (0.004) & (0.012) \\ 
  & & & \\ 
 political experience & 0.038 & 0.0003 & 0.007 \\ 
  & (0.034) & (0.010) & (0.030) \\ 
  & & & \\ 
 exp. private sector & $-$0.012 & 0.003 & 0.034$^{*}$ \\ 
  & (0.022) & (0.007) & (0.019) \\ 
  & & & \\ 
 exp. public administration & $-$0.030 & $-$0.003 & 0.009 \\ 
  & (0.023) & (0.007) & (0.020) \\ 
  & & & \\ 
 exp. non-profit org. & 0.050$^{**}$ & $-$0.013$^{*}$ & 0.042$^{**}$ \\ 
  & (0.023) & (0.007) & (0.020) \\ 
  & & & \\ 
 academic experience & 0.107$^{***}$ & 0.006 & $-$0.089$^{***}$ \\ 
  & (0.023) & (0.007) & (0.020) \\ 
  & & & \\ 
 SMD loose election x reelection & $-$0.014 & 0.004 & $-$0.065 \\ 
  & (0.068) & (0.021) & (0.059) \\ 
  & & & \\ 
 PR x reelection & $-$0.056 & 0.005 & 0.019 \\ 
  & (0.065) & (0.020) & (0.057) \\ 
  & & & \\ 
 Constant & $-$0.015 & 0.100$^{***}$ & 0.354$^{***}$ \\ 
  & (0.049) & (0.015) & (0.042) \\ 
  & & & \\ 
\hline \\[-1.8ex] 
\textbf{Congress controls} & Yes & Yes & Yes \\ 
\textbf{Party controls} & Yes & Yes & Yes \\ 
Observations & 2,499 & 2,454 & 2,308 \\ 
Adjusted R$^{2}$ & 0.183 & 0.004 & 0.042 \\ 
\hline 
\hline \\[-1.8ex] 
\textit{Note:}  & \multicolumn{3}{r}{$^{*}$p$<$0.1; $^{**}$p$<$0.05; $^{***}$p$<$0.01} \\ 
 & \multicolumn{3}{r}{Standard Errors in parentheses} \\ 
\end{tabular} 
\end{table} 

\begin{table}[!htbp] \centering 
  \caption{Effect of Electoral Reform and Electoral Institution over Legislative Effort: Productivity Measures} 
  \label{} 
\begin{tabular}{@{\extracolsep{5pt}}lcccc} 
\\[-1.8ex]\hline 
\hline \\[-1.8ex] 
 & \multicolumn{4}{c}{\textit{Dependent variable:}} \\ 
\cline{2-5} 
\\[-1.8ex] & Floor Attendance & Committee Participation & Floor Speech & Bills introduced \\ 
\\[-1.8ex] & (1) & (2) & (3) & (4)\\ 
\hline \\[-1.8ex] 
 SMD & 0.017$^{**}$ & 0.041 & $-$0.285 & $-$1.347$^{**}$ \\ 
  & (0.009) & (0.056) & (1.572) & (0.667) \\ 
  & & & & \\ 
 reelection & 0.126$^{***}$ & $-$0.078 & 4.047 & 2.919$^{**}$ \\ 
  & (0.018) & (0.114) & (3.223) & (1.368) \\ 
  & & & & \\ 
   SMD x reelection & $-$0.014 & 0.027 & $-$6.477$^{*}$ & $-$0.778 \\ 
  & (0.019) & (0.121) & (3.436) & (1.458) \\ 
  & & & & \\ 
 female & 0.002 & 0.077 & $-$1.818 & 0.868 \\ 
  & (0.008) & (0.049) & (1.398) & (0.593) \\ 
  & & & & \\ 
 education & 0.032$^{***}$ & 0.091$^{***}$ & 2.190$^{**}$ & 0.613$^{*}$ \\ 
  & (0.005) & (0.030) & (0.862) & (0.366) \\ 
  & & & & \\ 
 political experience & 0.040$^{***}$ & 0.033 & $-$2.994 & $-$0.169 \\ 
  & (0.012) & (0.076) & (2.166) & (0.919) \\ 
  & & & & \\ 
 exp. private sector & 0.016$^{**}$ & 0.082 & $-$4.325$^{***}$ & $-$1.343$^{**}$ \\ 
  & (0.008) & (0.051) & (1.451) & (0.616) \\ 
  & & & & \\ 
 exp. public administration & 0.002 & $-$0.122$^{**}$ & 0.023 & 0.211 \\ 
  & (0.008) & (0.052) & (1.482) & (0.629) \\ 
  & & & & \\ 
 exp. non-profit org. & 0.034$^{***}$ & 0.035 & $-$0.418 & 0.155 \\ 
  & (0.008) & (0.051) & (1.455) & (0.618) \\ 
  & & & & \\ 
 academic experience & 0.023$^{***}$ & $-$0.006 & 7.009$^{***}$ & 2.635$^{***}$ \\ 
  & (0.008) & (0.052) & (1.476) & (0.626) \\ 
  & & & & \\ 
 Constant & 0.620$^{***}$ & 2.655$^{***}$ & 20.328$^{***}$ & 13.827$^{***}$ \\ 
  & (0.035) & (0.222) & (6.275) & (2.663) \\ 
  & & & & \\ 
\hline \\[-1.8ex] 
\textbf{Congress controls} & Yes & Yes & Yes & Yes \\ 
\textbf{Party controls} & Yes & Yes & Yes & Yes \\ 
Observations & 2,500 & 2,500 & 2,500 & 2,500 \\ 
Adjusted R$^{2}$ & 0.148 & 0.391 & 0.128 & 0.118 \\ 
\hline 
\hline \\[-1.8ex] 
\textit{Note:}  & \multicolumn{4}{r}{$^{*}$p$<$0.1; $^{**}$p$<$0.05; $^{***}$p$<$0.01} \\ 
 & \multicolumn{4}{r}{Standard Errors in parentheses} \\ 
\end{tabular} 
\end{table} 


\end{document}